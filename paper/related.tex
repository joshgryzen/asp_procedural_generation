\section{Related Work}

\cite{SmithM11_Maze} present a general methodlogy of how ASP can be used to generate high-level overview design spaces within games, showcasing a chromatic maze generator. 
% 
\cite{Smith2014ALA_Dungeons} explore how ASP can be used to generate heirarchical game environments. The authors build dungeon maps in two stages. The first stage is the high-level overview environment, which includes rooms and pasasges for each 6 by 6 region. The second phase then breaks down the regions into 60 by 60 grids, placing specifc items and monsters. The authors found this approach to be computationally taxing, taking seconds to generate. 
% 


We try to combine these phases into one high level overview, providing a legal pathway between islands while assigning enemies, keys, and doors. 
% 
We leave the specifc placement of items wihtin the room to the game engine.
% 
No need for huge global grid - loading the entire map has a high overhead

Some other resources to read still: 

\begin{itemize}
    \item \href{https://ceur-ws.org/Vol-3204/paper_14.pdf}{An Application of ASP for Procedural Content
Generation in Video Games by Andrea De Seta, Mario Alviano}
    \item \href{https://scholarworks.calstate.edu/downloads/5712mc924}{DESIGNING A PROCEDURALLY GENERATED
METROIDVANIA STYLE VIDEO GAME USING
ANSWER SET PROGRAMMING by John Morris}
    \item \href{https://aaltodoc.aalto.fi/items/75ec3123-767b-46e1-af30-6c5740eab4fa}{Applying Answer Set Programming in Game Level Design by Antonova, Evgenia}
    \item \href{https://aircconline.com/ijaia/V14N3/14323ijaia02.pdf}{ROCEDURAL GENERATION IN 2D METROIDVANIA GAME WITH ANSWER SET PROGRAMMING AND PERLIN NOISE by John Xu1, John Morris2}
\end{itemize}